\section{Game description} \sectionauthor{N. Thykier}
This section will cover the game representation, its rules, the
(un)implemented features, and the technical challenges of making the
game.

\subsection{Game basics}
The game world is represented as a grid.  Each tile is basically
either a:

\begin{itemize}
\item Field - passable
\item Wall - impassable
\item Pressure plate (button) - passable
\item ``One-time pressure plate'' (button) - passable
\item Gate - state-dependent
\item ``One-time pass'' - state-dependent
\end{itemize}

There are two special singleton fields, the ``time machine'' and the
``goal''.  Both fields are passable like a normal field.  However,
they have special game semantics, which will be covered in the game
rules.

There are two movable elements, namely the player and crates.  The
former is directly controlled by the player and the latter can be
pushed by the player.

The game is a single player game, but due to the ``time machine''
there can be more than one ``player self'' present on the game board.
All ``selves'' are identical in terms of play rules (e.g. any of
them can pick up the goal).  Particularly, all ``selves'' must
eventually enter the time machine.  The only difference is that the
player actively controls the ``current self''.

Originally, these ``selves'' were referred to as ``time clones'' (or
simply just ``clones'').  Particularly, in the source code they are
still called ``clones'' and we may use the terms interchangeably.

The action that causes a new ``player self'' to appear is called a
``time jump'' and is explained in detail in the \ref{time-jumping}
section.

Some fields have multiple states and behaviours.  Please refer to
the \ref{field-states} section for more information.

A screenshot of the graphical representation of the elements is available
in figure~\ref{fig:IconExplanation}.

\figurepng{IconExplanation}{A visual representation of the game elements.}

\paragraph{Game objective}
To complete a given level, the player must obtain the goal (``gold coin''),
return to the time machine and enter it.

\subsection{Player actions}
\label{player-actions}
Each turn, the player can do one of the following moves:

\begin{itemize}
\item Stand still (i.e. ``do nothing'')
\item Attempt to move to a neighbouring square (i.e. up, down, left or right)
\item Enter time machine (only allowed on top of the time-machine)
\end{itemize}

The player can push a crate by moving ``into'' it, provided that the
destination for the crate is available.  It is not possible to
``pull'' crates and the player can push at most one crate at a time.
These are basically the same rules used in Sokoban for moving crates.

Any number of ``player selves'' are allowed to occupy the same
tile.

\paragraph{Time jumping}
\label{time-jumping}
The player has the ability to go back in time to the first turn.  As
the player goes back in time, the world will reset itself to its
initial state, but with one crucial difference.

There is now an extra ``player'' in the level; namely the ``current
self'' that performed the time jump and the ``past self/selves''.  The
latter will carry out their actions (as the player did in that time
jump) and eventually enter the time machine.

As mentioned earlier, all ``selves'' are identically in terms of play
rules (e.g. any of them can pick up the goal), except that only the
``current self'' is controllable by the player.  The ``past selves''
carry out the actions the player did when that ``self'' was the
``current'' one.

Time jumping allows the player to do things that would otherwise not
be possible by playing ``ping-pong'' with his/her ``past selves'',
though at the risk of introducing temporal paradoxes (see
section \ref{temporal-paradox}).

\subsubsection{Field states and manipulation}
\label{field-states}
Certain fields in the game world have multiple states with different
behaviours.  State changes are generally triggered as a result of the
player actions.  The following fields can change during the play:

\begin{itemize}
\item (Regular) pressure plate (button)
\item One-time pressure plate (button)
\item Gate
\item One-time pass
\end{itemize}

The gates start as either ``open'' (passable) or ``closed''
(impassable).  They alter state when a button activate them.  If
multiple buttons activate the same gate, then its resulting state
depends on the number of gates.  If the number is ``even'', the
gate is in its starting state and if the number is odd it is in the
``opposite'' state\footnote{Basically a ``boolean xor'' for those
  familiar with that term.}.

All buttons are activated when a player or crate is on top of them.
The difference between a one-time pressure plate and a regular
pressure plate is that a regular activated pressure plate will
deactivate when nothing is on top of it (like a ``dead man's
switch'').  A one-time pressure plate can, once pressed, never be
``unpressed'' in the ``future".

A one-time pass (short for ``one-time passage'') is a field that
(as the name suggests) can only be passed once.  The passage will
``collapse'' on the turn after a player or crate enters it.  From a
semantic point of view, the field turns into a wall.  The net effect
is that a player cannot linger on the field and crates cannot be pushed
on to the field\footnote{Because there is no way to push it off the field
and it is not allowed to stay on the field either.}.

As the keen reader may already have guessed (or noticed), ``one-time''
is defined as ``once per time jump'' and not ``once per level''.  When
performing a time jump, these fields are still reset to their initial
state (See \nameref{time-jumping}).


\figurepng{button-gate-ex}{An example with a pressure plate and a closed gate.}

\subsubsection{Losing}

There are three different ways of losing
\footnote{Implementation Detail: The game calls all cases for a ``Time paradox''.}.

\paragraph{Getting stuck}
If a ``player self'' gets stuck and is unable to return to start, it
is not possible to win.  If the player gets stuck in a closed gate or
a crate, the game will automatically mark the game as lost.

If the current ``player self'' simply manages to ``lock
himself/herself into a corner'' with no path back to start, the game
will still continue (despite being unsolvable) until the player realises
the mistake, restarts or gives up.

\paragraph{Temporal paradox}
\label{temporal-paradox}
As the name suggests, a temporal paradox (``time paradox'') is a bad
thing.  What happens if your current self prevents your past self from
becoming the current self\footnote{This problem is known as the
  ``Grandfather paradox'.'}?  In this game, you simply lose.

\paragraph{Non-Determinism}
All valid actions are carried out ``simultaneously''.  If one (or
more) valid action(s) causes another valid action to fail, the result
is non-deterministic.

As an example, consider if two clones attempt to push the same
crate.  This gives 3 cases, of which one is non-deterministic.

\begin{itemize}
\item Both clones pushes it in the same direction (i.e. they perform
  the same action and start on the same field), the actions are
  deterministic and will be carried out.
\item Both clones pushes it in opposite directions.  In this case,
  both actions are ``invalid'' because each clone can conclude (before
  pushing) that the move will not work.  This is deterministic and
  result in none of the clones moving.
\item The clones pushes it in ortogonal directions.  This causes
  non-determinism as each clone believe the action is valid, but
  both actions cannot be carried out.
\end{itemize}

This rule is (also) a generalization of getting stuck in a gate or
crate. Non-determinism always results in losing.

\subsubsection{Scoring}
The player is scored on the number of moves performed and the number
of time jumps used.  Internally, the game uses an absolute counter,
where ``lowest'' is considered best (i.e. most efficient).
This counter is changed according to the rules below:

\begin{itemize}
\item +1 per move or skipped turn by the ``current self''.
  Entering the time machine counts as a move.
\item +1 per skipped turn while the ``current self'' is outside
  the time machine.  Though, 0 if the ``current self'' is inside the
  time machine.
\item +1 for every time jump performed.
\end{itemize}

As our play testers generally ignored it, we tried to represent it
differently.  We deviced a ``two-step'' formular to do this.  If $x$
is the counter defined above, then the score is presented as:

\[
f(x) =
\begin{cases}
100 - x \cdot \frac{2 - \frac{x}{500}}{5}, & \mbox{if } 0 \leq x < 450\\
\frac{1}{x - 449}, & \mbox{if } x \geq 450
\end{cases}
\]

The formula was the result of some trial and error around the following
criteria:

\begin{itemize}
\item The score should decline monotonously.
\item The score should not become negative.
\item Two (reasonable) solutions should be trivially distinguishable by score.
\item Improving a good solution with 1 step should give more points
  than improving a poor one with 1 step.
\end{itemize}

The function (almost) satisfies all of the above points.  The fourth
is slightly violated around 450 turns, when the defintion cases
changes.  When x goes from 450 to 451, the score drops by a half,
which is the highest drop in its entire definition.

\subsection{Desired features}
We wanted quite a few features; most of these are fields or game
mechanics and are listed below.

\begin{itemize}
\item[+] Time machine allowing the player to do time-jumps.
\item[+] Two different buttons (pressure plates)
\item[+] XOR gates.
\item[+] Pushable crates.
\item[+] One-time passable fields.
\item[-] Activatable Teleporter (field).  A field that can be
  activated/disactivated.  When activated it would teleport anything
  entering it to another field.  When deactivated, it would behave
  like a regular field.
\item[-] Crate destroying field.  It would ``remove'' any crate pushed
  on top of it.  The crates would only re-appear if the player did a
  time-jump (as usual).
\item[*] Gates requiring more than one button to be activated/deactivated.
 (Emulation: Multiple gates)
\item[*] Gates that remain activated as long as at least one button activates
 it.  (Emulation: Multiple gates)
\item[-] One-way fields.  Idea is that the field forces the player to go
 in a certain direction.  Possibly activateable to have it rotate either
 clockwise or counter-clockwise.
\item[-] Buttons (pressure plates) only activateable by crates.
\item[-] Limit on the number of moves per clone or/and number of time-jumps.
\item[+] Group levels in ``campaigns''.
\item[-] Limit the number of moves/time-jumps across all levels in a campaign.
\end{itemize}

Beyond that we also have a list of features that improves the game
experience without affecting the game rules.

\begin{itemize}
\item[+] Sounds and music (that can be muted).
\item[+] Fully functional level editor.
\item[+] Instructions (``How to play'') available from the game.
\item[+] Snow theme.
\item[+] Reset level to start.
\item[+] Reset (undo) current clone
\item[+] Recorded (and replayable) solution to levels.
\item[+] Automated regression test suite for the game logic.
\item[-] Undo last move.
\item[-] Undo more than one clone.
\item[-] Displaying the actions of each clone (e.g. as arrows/small icons).
\item[-] Hilighting the ``current self'' (e.g. with an arrow).
\item[-] Showing a combined score.
\item[-] Interactive tutorial
\end{itemize}

\paragraph{Legend}
\begin{itemize}
\item[-] Not implemented
\item[+] Implemented
\item[*] Not implemented, but can be emulated ``trivially'' from existing
 features.
\end{itemize}

\subsection{Technical challenges}
There were some major technical challenges.

The primary technical challenge was actually the levels.  Finding a(n
optimal) solution to a Sokoban level is already known to be NP-Hard.
Adding a time machine and extra game mechanics on top of that is
unlikely to make it any easier to find an optimal solution.

We did not have any mechanically way of finding the solution to any of
our levels.  But more importantly, we had no way of verifying that the
solution we found was indeed the optimal one.  Especially, we had no
way of checking if our solution was (still) optimal if the scoring rules
were changed.

Another perceived issue was detecting the losing criteria.  While it
turned out to be fairly easy in practise, we did spent quite some time
figuring out all the cases that could introduce non-determinism and
how to detect them.

There is also the issue of detecting if the ``current self'' is
stuck in a part of the level.  While there are some trivial cases,
that would be ``easy'' to detect, in general it will probably be
just as hard as solving the level.

Argument: Consider a hypothetical level where the ``current self'' can
make it back to start.  But to do so, the player have to move a number
of crates on top of regular pressure plates to open just as many
gates.  This is basically solving a Sokoban level and thus just as
hard as doing so.

