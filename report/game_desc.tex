\section{Game description}

This section will cover the game representation, its rules, the
(un)implemented features, and the technical challenges of making the
game.

\subsection{Game basics}
The game world is represented as a grid.  Each tile is basically
either a:

\begin{itemize}
\item field - passable, stateless
\item wall - impassable, stateless
\item button - passable, source (activator)
\item gate - passible/impassable, target (activatable)
\end{itemize}

There are two special singleton fields, the ``time machine'' and
the ``goal''.  Both fields are passable like a normal field, but
they have special game semantics, which will be covered in the
game rules.

There are two movable elements, namely the player(s) and crates.  The
former are directly controlled by the player and the latter can be
pushed by the player.

A screenshot of graphical representation of the elements is available
in figure~\ref{fig:IconExplanation}.

\figurepng{IconExplanation}{A visual representation of the game}

\paragraph{Game objective}
To complete a given level, the player must obtain the goal (``gold coin''),
return to the time machine and enter it.

\subsection{Player actions}
Each turn, the player can do one of the following moves:

\begin{itemize}
\item Stand still (i.e. ``do nothing'')
\item Attempt to move to a neighbouring square (i.e. up, down, left or right)
\item Enter time machine (only allowed on top of the time-machine)
\end{itemize}

The player can push a crate by moving ``into'' it provided that the
destination for the crate is available.  It is not possible to
``pull'' crates and the player can push at most one crate.  These
are basically the same rules used in Sokoban for moving crates.

\paragraph{Time jumping}
The player has the ability to go back in time to the first turn.  As
the player goes back in time, the world will reset it self to its
initial state, but with one crucial difference.

There is now an extra ``player'' in the level; namely the ``current
self'' that performed the time jump and the ``past self/selves''.  The
latter will carry out their actions (as the player did in that time
jump) and eventually enter the time machine.

All ``selves'' are identically in terms of play rules (e.g. any of
them can pick up the goal).  Particularly, all ``selves'' must
eventually enter the time machine.  The only difference is that the
player actively controls the ``current self''.

Time jumping allows the player to do things that would not otherwise
be possible by playing ``ping-poing'' with his/her ``past selves'',
though at the risk of introducing temporal paradoxes (see
\nameref{temporal-paradox}).

Any number of ``player selves'' are allowed to occupy the same
tile.

\subsubsection{Buttons and gates}
The player can also manipulate the game world via buttons (an
``activator''), which then triggers a gate (an ``activatable'').

Buttons are ``on'' when either a create or at least one ``player
self'' is on top.  While the button is ``on'' it will keep its
``activatables'' active.  As an example, consider the little game
world in figure~\ref{fig:button-gate-ex}.

\figurepng{button-gate-ex}{A small example game with a button and a closed gate}



\subsubsection{Losing}

There are three different ways of losing
\footnote{Implementation Detail: The game calls all cases for a ``Time paradox''.}.

\paragraph{Getting stuck}
If a ``player self'' gets stuck and is unable to return to start, it
is not possible to win.  If the player gets stuck in a closed gate or
a crate, the game will automatically mark the game as lost.

If the player simply manages to ``lock himself/herself into a corner''
with no path back to start, the game will still continue (despite being
unsolvable) until the player restarts/gives up.

\paragraph{Temporal paradox}
\label{temporal-paradox}
As the name suggests, a temporal paradox (``time paradox'') is a bad
thing.  What happens if your current self prevent your past self from
becoming the current self?  In this game, you simply lose.

\paragraph{Non-Determinism}
All valid actions are carried out ``simultaneously''.  If one (or
more) valid action(s) causes another valid action to fail, the result
is non-deterministic.

As an example, consider if two clones attempt to push the same
crate.  This gives 3 cases, of which one is non-deterministic.

\begin{itemize}
\item Both clones pushes it in the same direction (i.e. they perform
  the same action and start on the same field), the actions are
  deterministic and will be carried out.
\item Both clones pushes it in opposite directions.  In this case,
  both actions are ``invalid'' because each clone can conclude (before
  pushing) that the move will not work.  This is deterministic and
  result in none of the clones moving.
\item The clones pushes it in ortogonal directions.  This causes
  non-determinism as each clone believe the action is valid, but
  both actions cannot be carried out.
\end{itemize}

This rule is (also) a generalization of getting stuck in a gate or
crate.

\subsubsection{Scoring}
The player is scored on the number of moves performed and the number
of time jumps used.  For comparsion, the lowest score is considered
best (i.e. most efficient).  The scoring rules are:

\begin{itemize}
\item 1 point per move or skipped turn by the ``current self''.
  Entering the time machine counts as a move.
\item 1 point per skipped turn while the ``current self'' is outside
  the time machine.  0 points if the ``current self'' is inside the
  time machine.
\item 1 point for every time jump performed.
\end{itemize}

\subsection{Desired features}

\subsection{Technical challenges}

