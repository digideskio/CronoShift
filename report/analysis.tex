\section{Game analysis}
This section is devoted to an in-depth quality-in-use assessment of
Jikiban, where the basic theory and concepts will be lifted from
several articles taking a scientific approach to game design theory.

\subsection{Playability}
In \cite{sanchez09} the concept ``playability'' is defined as:

\begin{quotation}
The degree to which a specific player achieve specific game goals with
effectiveness, efficiency, flexibility, security and, especially,
satisfaction in a playable context of use.
\end{quotation}

The layout will first delineate the different concepts used and then
proceed to an analysis of how these concepts helped shape and define
Jikiban as an interactive entertainment system.


\subsection{Playtests}
\label{sec:playtests}

Three iterations of playtests were carried out, with interleaved
design changes to the game. The first of these iterations was
initiated as soon as a working game was up and running. This section
will not go through all individual iterations, but will briefly
outline the most noteworthy findings.  Early playtests showed that
the game was largely unplayable. It should be noted that this early
version had significantly fewer features than the final version. Most
noteworthy

\begin{itemize}
\item Although the levels were solvable, they had never been playtested.
\item There were no hints at all, so the players had no help at all.
\end{itemize}

The players had trouble understanding the concepts of the game such as
how the wait function works in the context of multiple player
copies. Only the active character waits.  However, if the concept of the
past-selves is not well understood, this is not trivial. It becomes
odd that the other past-selves move regardless of what the active
character does. This problem was addressed through a change in the
slope of the learning curve in early levels.

The time machine was the hardest concept to understand. Since the
character was standing on it when entering, it was not apparent that
time had been reset. Now there were simply two versions of the player,
one of which was very hard to control (actually uncontrollable). Both
of these concepts were introduced in level 2 along with buttons and
gates, so the player basically gets overwhelmed early in the game and
gives up.

The player often had problems identifying the active player. There
were nothing to mark which player was actually being moved and the
other past-selves were hidden when standing on the same spot. The
result was that the player's timing was ruined forcing him to restart,
due to a mistake.

The enormous amount of counting required to get through a level often
deterred the players from completing it at all. Often they would find
the trick to the level and then skip it, since the rest was just dull
work. A player described it as having to program the level after
finding the solution. This issue was addressed using patient music,
sound effects on level completion and a score system which encourages
optimization of solutions.

Changes were made to give the player more help in the form of hints,
the hint was displayed below each level. This resulted in a complaint
that the hint was always visible and gave away the solution. On the
bright side the game could now be played to the end, without help, by
experienced gamer. The player would not have played it this much if he
hadn't been asked to though. The game lacked fun and a motivation
factor.

%TODO: align with Evolution part
To accommodate these problems, the hint was hidden behind a button.
Some of the levels were rewritten to incorporate one time buttons and make
a more smooth level progression using these. This enabled the gates
and one-time buttons to be introduced before the ability to go back in
time.
