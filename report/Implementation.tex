\section{Implementation}
\subsection{Unity implementation}
The game has initially been developed in unity, which is the default development tool for the course, but for technical reasons it was rewritten in python. This section decribes the inital steps of the development cycle before this change.
Initially the game was supposed to be a fusion between the two games Chronotron and Shift. Shift is a 2D platform game seen from the side where the world is black and white, which can be inverted at the touch of a button. Our game is seen from the top, so this feature is not trivial to implement, and was therefore posponed until the other core mechanics where in place.
The mechanics were divided into four scripts:\\

\begin{itemize}
\item Manager: responsible for loading and running the game.
\item Turn controller: which moves all the clones each turn and checks for possible paradoxes.
\item Player: the script controlling the movement of the player character as well as reading user inputs.
\item Past self: which remembers and replays the players previous movements and animates it.
\end{itemize}

The first step was to get a movable player, which was done by importing the thirdperson controller script from unity. The player could now be moved around the plane but could not be rotated and more importantly the movement was in real time. If the game was going to be turnbased the player needed to move the same distance each time, so the user could predict the outcome of the movement. A few attempts were made to addapt the thirdperson controller, but in the end it was completely rewritten, since most of it was not needed anyway.

\paragraph {Level generation} At this point the world was still a flat plane, so work began on the manager whose responsibility included loading an actual level. The level is made up of empty tiles walls and interactive objects. To simplify creation of new levels a simple text file format was chosen, which the manager interprets and make a graphical representation of. This way the levels do not need to be hard coded a all and anyone can potentially make new challenges. This format was chosen for the file \ref{tab:game-object-representation}.

\begin{table}
\begin{tabular}{|c|c|}
\hline
Char & Game object\\
\hline
+ & Wall\\
`` '' & Field\\
b & Pressure plate\\
$\_$ & Gate (open)\\
- & Gate (closed)\\
S & Start/Time machine\\
G & Goal\\
c & Crate\\
o & One-time button\\
p & One-time pass\\
\hline
\end{tabular}
\label{game-object-representation}
\caption{Game object representation}
\end{table}

\begin{table}
\begin{verbatim}
2D SuperFun!
++++++
+Go- +
+-++ +
+ ++ +
+_S o+
++++++

button (2, 1) -> gate (3, 1)
button (4, 4) -> gate (1, 2)
button (4, 4) -> gate (1, 4)

Description: If you are stuck, return to start.
Solution:
 EE WW T
 .
 WN NN EE ES SS WW T
\end{verbatim}
\label{example-level-file}
\caption{Example level file}
\end{table}

In the level format \ref{tab:example-level-file}, the first line is simply a magic word (the game didn't have
a title at this point). After that is the level layout and the mapping between
the interactive game obejcts. The decription gives a hint about hot to solve
the level. The solution is a series of commands that enables the computer to
play through the level on its own. The solution was not a planned feature at
this point in the development, but it is a part of the final file format.

\paragraph{Unity problems}
While it is farily simple to restrict unity to a top-down 2D view, it
is not so trivial to restrict it to a turnbased format. The update
fuction was used to move the player and the ``past selves", but they need
to move at the exact same time and exactly one tile.

With one player the controls can be locked, so that no movement commands
are accepted, before he has moved to the middle of the next tile. When
there is more than one player, the computer has to move the remaining player
characters. But now locking the controls doesn't solve the problem,
because the active player is not the only one, who needs to finish
moving. If the player moves before all his past selves and crates have
been moved, the game becomes desynchronized and the game rules can be
broken. This can partly be solved by making sure that all clones have
finished, by having all of them notify the turn controller. This make
the turncontoller and the player script tightly coupled however.

\subsection{Pygame implementation}
After the file format was established, the need arose to check whether
a level file was correct. This required a lot of logic including
implementation of all the game rules. But if the rules were already there
it was fairly simple to remake what had already been made in unity.

Pygame and PGU was used as a basis for the GUI, the actual icons were
also burrowed from this framework, though the majority is self made as
described in the LICENSES.IMAGES file.

Pygame had the advantage that it was much easier to restrict to a
turnbased format, because of an less hardcoded update function. The
animations where later improved to run in real time while the inputs could 
be accepted and processed as fast as the user could type. The game state
would update after each input, while the animation evatually caught up.

We used the model-view-controller structure to seperate the three
components. This meant that we had more control over when the view
updated, so it could run independent of the controller. The controller
handles all the inputs and updates the model. The view is quite passive
and only updates when the models sends an event, usualy as a result
 of some request from the controller. Unity does not seperate view
 and controller, which caused problems for us.

\paragraph{Adding solutions}
to the levels was one of our desired features. With the logic in place,
the game could be extended to play itself, given a sequence of directions.
While this might not sound very useful for a single-player game, it enabled
us to check a solution by having the computer repeat it. This also meant,
that a user could get the correct solution ingame, if they gave up.

\figurepng{Editor}{The in-game editor}
\paragraph{The editor} is one of the major components of the game.
Because the level is just a text file that the program interprets, new
levels can easily be introduced or changed. The editor provides an
environment for people who does not know the code to make their own
levels.The editor can be seen in figure\ref{fig:Editor}.

The editor works by loading all fields the game includes into the
buttom bar, where one can be selected and placed. If the placed field
is the time-machine or the goal the program automatically removes any
previous instance of this field from the level.  This is to ensure the
level is semantically correct.

But a level also includes a mapping section between buttons and gates,
which are the only fields requiring a mapping to another field. This
proved a bit more difficult to make then simply placing fields. The
solution is a ``none'' button, indicating that nothing is
selected. This enabled the user to right click on existing fields
without overriding them, and left click another field to map it to.

The last part, the solution, is added by simply playing the level to the end
and saving it to the level in the editor with F2. The final result can be saved
in the editor. There is currently no way to add a hint from the editor.

\paragraph{Providing help} to the users has driven a lot of the development.
The first playtests\ref{sec:Play-tests} showed that the game was to difficult,
or at least hard to understand. Therefore a hint button was added to give
hint for how to solve the level, but that didn't really explain the game
rules or the controls. Remaking all levels improved this, though only to a
certain degree. A help menu was added as a seperate window explaining
how to use every aspect of the game. Users rarely read the manual, but if
all else fails they should at least have the oppotunity. The help window is
shown in figure\ref{fig:HelpMenu}.

\figurepng{HelpMenu}{Help menu}

