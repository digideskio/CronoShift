\section{Dicussion} \sectionauthor{N. Thykier}

\subsection{Evolution}
The original pitch started as ``Mix the time-jump feature from
Chronotron with the Shift feature from shift''.  The game ended up as
something more like ``Take the time-jump feature from Chronotron and
add some Sokoban crates''.  In this section we will describe how and why
we changed the pitch over time.\\


When we started, we looked at doing a mix of Chronotron and Shift.  To
keep things simple, we decided to reduce the game to a turn-based
variant (both Chronotron and Shift are real-time games).

Using a grid world as basis, we could fairly trivially reason about
the time machine mechanics.  In this initial design, we realised that
without some manipulatable obstacles the time machine would be
redundant.  We created the ``gate'' and the ``pressure plate'' for
this purpose.

But we could never really agree on how to the ``Shift'' feature should
work in this game world.  After a few weeks we ended up shelving it as
we felt it was just getting in our way.  We decided to add the
pushable crates from Sokoban as a replacement.  These crates turned out
to be very hard to use in the design of levels

In Sokoban, crates have to end at specific nodes (``goal locations'').
But in Jikiban, the player is by default not required to do anything
with a crate.  Due to rules behind how to push crates, the player
could usually just walk around the crate.  Also, with a time-machine
the player can always create a time-clone to replace the crate when
working with a regular pressure plate.

We thought about doing a pressure plate that could only be triggered
by a crate.  However, before implementing that field, we deviced a way
to use crates to emulate a one-time pressure plate in the game.  This
also lead to a one-way passage and the ``suicide''-construction in the
level called ``tutorial10.txt'' (see figure \ref{fig:suicide-lvl}).

%TODO: align with Play-test part
After a while, the one-time pressure plate emulation got annoying to
redo in each level.  We implemented the actual one-time pressure
plate. We also added the ``one-time passage'' to compliment it.  It
also simplified a number of levels as a one-time pressure plate was
now a single field (instead of a 3-4 fields surrounds by gates and
walls).

\figurepng{suicide-lvl}{Level ``tutorial10.txt'': If you push any of
  those crates ``out of the way'', they will activate the gate below
  you.}


\subsection{If we could do it all over \ldots}
The first thing would have been not wasting any time with Unity.
While Unity is (probably) a reasonable prototype framework for
real-time position-based games, it felt very cumbersome for turn-based
and node-based games.

Secondly, we would seriously consider designing the game with a level
editor in mind from the start.  Originally levels were written by
``hand'' in a regular text editor.  For anything but the smallest
levels it gets very difficult to envision the level and also counting
the offset gets tedious very quickly.

Make it easier to get the solutions from players.  Some of our play
testers have found some ``interesting solutions'', but
unfortunately we are unable to get the recorded solutions.  While we
could reason about the efficiency of the solution being inferior to
ours, seeing how the player solved the level might have proven useful
to us.

